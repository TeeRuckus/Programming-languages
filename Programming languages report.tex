\documentclass[
	12pt, % Default font size, values between 10pt-12pt are allowed
	%letterpaper, % Uncomment for US letter paper size
	%spanish, % Uncomment for Spanish
]{fphw}

% Template-specific packages
\usepackage{color} %red, green, blue, yellow, cyan, magenta, black, white
\usepackage[utf8]{inputenc} % Required for inputting international characters
\usepackage[T1]{fontenc} % Output font encoding for international characters
\usepackage{mathpazo} % Use the Palatino font
\usepackage{graphicx} % Required for including images
\usepackage{subfig}
\graphicspath{{figures/}}
\usepackage{array}
\usepackage{amsmath}
\usepackage{amssymb}
\usepackage{booktabs} % Required for better horizontal rules in tables
\usepackage{listings} % Required for insertion of code
\usepackage{enumerate} % To modify the enumerate environment
\usepackage[section]{placeins}
\usepackage{wrapfig}
\usepackage{pdfpages}
\usepackage[section]{placeins}
\usepackage{flafter}
%trying to do referencing using Chicago style
%NEW SHIT HERE
\usepackage{appendix}
%delete after, this is to place dummy text on the page
\usepackage[english]{babel}
\usepackage{setspace}
\usepackage{csquotes}
%how to use this package. Do your code snippets last, as you need to move the .m file into
%the same directory as this tex file

%adding a box around my imported latex code
\definecolor{mygreen}{RGB}{28,172,0} % color values Red, Green, Blue
\definecolor{mylilas}{RGB}{170,55,241}\lstset{frame=single, rulesepcolor=\color{black}, numbers=left}

\lstset{language=Matlab,%
    %basicstyle=\color{red},
    breaklines=true,%
    morekeywords={matlab2tikz},
    keywordstyle=\color{blue},%
    morekeywords=[2]{1}, keywordstyle=[2]{\color{black}},
    identifierstyle=\color{black},%
    stringstyle=\color{mylilas},
    commentstyle=\color{mygreen},%
    showstringspaces=false,%without this there will be a symbol in the places where there is a space
    numbers=left,%
    numberstyle={\tiny \color{black}},% size of the numbers
    numbersep=9pt, % this defines how far the numbers are from the text
    emph=[1]{for,end,break},emphstyle=[1]\color{red}, %some words to emphasise
    emph=[2]{word1,word2}, emphstyle=[2]{style},
}

\usepackage[authordate]{biblatex-chicago}
\DeclareFieldFormat[article]{title}{\mkbibquote{#1}} % make article titles in quotes
\DeclareFieldFormat[thesis]{title}{\mkbibemph{#1}} % make theses italics
\bibliography{Programming languages report}

\counterwithin*{equation}{section}
\counterwithin*{equation}{subsection}
\counterwithin*{equation}{subsubsection}

%CHANGE
%I want to save space when I am writing my document, so I am going to have
%everything as the default values for the system
%\setlength{\parindent}{0pt}
\setlength{\parskip}{1.0em}
%\onehalfspacing
%\setstretch{1.25}
%making a variable for where all my results are going to be stored
\newcommand\pathFiles{"summary/"}
%----------------------------------------------------------------------------------------
%	ASSIGNMENT INFORMATION
%----------------------------------------------------------------------------------------

\title{Programming Languages Assignment} % Assignment title
\author{Tawana Kwaramba: 19476700} % Student name
\date{October 25th, 2021} % Due date
\institute{Curtin University \\ Faculty of Science and Engineering: School of Electrical Engineering, computing and Math Science} % Institute or school name
\class{Programming Languages - COMP2007} % Course or class name
\professor{Ascsociate Lecturer: Arlen Brower}
%----------------------------------------------------------------------------------------

\begin{document}
\pagenumbering{gobble}
\maketitle
\newpage
\tableofcontents
\newpage
\listoffigures
%\newpage
\listoftables
\newpage
%\end{spacing}
\pagenumbering{arabic}
%----------------------------------------------------------------------------------------
%	Programming Languages Report
%----------------------------------------------------------------------------------------
\section{Introduction}

\section{Programme Testing and Programme execution}
Each programme folder in this assignment is going to have its own independent
\textbf{README.md}, and a \textbf{run.sh} file. Therefore, all the marker has to
do is to execute the run.sh file by typing \emph{./run.sh} and that will demonstrate
the functionality of my programme. It should be noted that practical
four doesn't contain a run.sh file instructions outlining on how the scripts
should be executed has being left in README.md.

\section{FORTRAN}
\subsection{Fortran: Discussion}
Fortran was based on the programming paradigm of the punch card system which imposed specific
rules on Fortran. Which included the following: the first column of each row is going to be reserved for
the comment character which is either a c or an asterisk ("c" or "*"); column one to five of each row
is going to be reserved for statements or labels; column six is going to be
reserved for the continuation of a command from the previous line; commands will
terminate on column seventy-two; and column seventy-three
to eighty are going to be reserved for sequence numbers. As a consequence, this
made Fortran more intellectually involving compared to previously written
languages of Java, C and Python. \par

Programming in Fortran I had to be more conscious on the current column number
which is typically not a behaviour I do while programming in the named languages.
In the named languages I would typically choose to keep each line below 80 characters
to make it easier for others to read the my code. Therefore, due to this imposed
rule Fortran's \emph{writability} is not like the named languages. \par

Fortran's variables is only limited to one-to-six characters long limiting the
expressivity of variables in Fortran. In some cases six characters is not enough
to fully explain the purpose of variable hence, compromises in variable naming
will have to be made. Compared to Java, C and Python the programmer can fully
express the purpose of variable as they is not limitation in variable name
length. Therefore, as demonstrated Fortran is less expressive in variable naming
as compared to Java, C and Python. Additionally, due to the limitation of the
variable length Fortran as well is going to break the \emph{zero-one-infinity}
programming principle. \par

Fortran doesn't support reserved words causing the change of behaviour of functions,
and resulting in a less \emph{reliable} programming experience. Programming in
Fortran requires double checking variable names to ensure that they were not
overriding in-built functions as the complier will not raise these incidents as
errors. As compared to the named languages the complier will raise the incidents
as errors, and the programmer typically would not have to concern themselves
with the naming of variables. Therefore, due to this Fortran will require the
programmer to involve themselves with behaviours which they're not accustomed too.
Additionally, since Fortran will allow the re-definition of in-build functions
meaning that in a programming project an intern can override a in-built function
to do something else and they would not know as the complier will not flag it i.e.
the do-while loop can be over ride to add one instead of looping. Hence, for this
reason programming in Fortran in less \emph{reliable} as compared to the named
languages.\par

Additionally, Fortran is less \emph{reliable} than Java, C and Python due the compilation
process. Fortran doesn't look for uninitialised variables hence, Fortran will
allow you to compile and run a programme even if the variable hasn't be declared
and not assigned to anything. Additionally, during execution the uninitialised
variable will be a random memory address thus, any operations in the programme
will be done to that specific memory address which in some cases can lead into
unintended actions resulting, in Fortran breaking the \emph{security} and
\emph{defense in depth} programming principle. As compared to the named languages
the complier will raise this incident as an \emph{uninitialised variable error}
therefore, stopping the user from accessing memory which they should not be accessing.
Therefore, as demonstrated Fortran in less \emph{reliable} and less \emph{secure}
than the names languages.\par

Similarities of Fortran as compared to the named programming languages is that
the first character of each variable has to start with a letter and can't start
with a number, Fortran will require you to declare the types of your variables
same as C, and Java. Furthermore, Fortran will require the programme fields to be
the same name as the current file name which is a similar idea which is seen in
java whereby the file name has to be the same as the class name, and Fortran will
require terminating statements for each command same as Java, and C.

\subsection{Fortran: reflection}
Fortran doesn't have scope as consequence it's difficult to plan and write
a full programme. As a result of no scope variables can be accessed from anywhere
causing unattended side effects as they is not protection from scope. Therefore,
this is going to make it harder to debug larger programmes, as the
programmer would have to have full knowledge of the programme instead
of knowledge of the current scope. Making Fortran difficult to structure code
in the appropriate hierarchies hence, violating the \emph{structured programming principle}.
Additionally, Fortran not allowing the hiding the implementation of abstract data
structures, and implementations violating the \emph{information hiding
principle}.\par


I found Fortran a lot harder to debug as the complier doesn't display helpful
messages. During this programming assignment I spent hours trying to figure
out why my programme was not working, although the logical structure of the
programme was correct. To only find out what it was due to that I had not
initialised a variable properly. Therefore, Fortran is not a language I would
personally use for large scale projects as the complier is not helpful. Additionally,
due to one small mistake breaking a programme Fortran is not \emph{reliable}.\par


Additionally, the complier will see all commands and variables as upper cases hence, making the language
case insensitive further adding on onto the \emph{unreliability} of the Fortran
programming language, as how would the programme know that they're going to be
accessing the correct variable or constant in the programme.\par
\section{ALGOL 68}
\subsection{ALGOL 68: Fizz-buzz Activation Record}

\subsection{algo 68: reflectoin}
writting the fizz buzz programme with algo 68 was more plesant than fortran. algo
68 is starting to represent the langauges which we're more accostumed too. alog 68
strongly reminds me of the bash scripting language as the constructs are going
to end with a word instead of terminating delimeter. \par

algo 68 is going to be a language which is going to be scopped hence, the
variables which are in a \emph{begin} and \emph{end} block are only going to be
found to be accessed within that block, and variables which are going to be defined
in any given construct are going to be only accessed within those constructs.
therefore in algo 68 you're not going to get the side effects whichy you may get in
fortran as the variables are going to be protected by scope therfore, algo 68
adhering to the \emph{information hiding principle}. \par

additionally, due to the introduction of scope, and the introduction of clear
termination of constructs, it's a lot easier to look at an algo 68 programme and
get a good idea on what the intededed purpose of the programme is going to be
hence, adhering to the \emph{readability principle}. since, the algo-68 is going
to be highly structured this also adheres to the \emph{structured programming}
principle. as a consquence to these observation, wrting a programme in algo-68
was a lot easier than writing a programme in fortran.

\section{ada}
\subsection{ada: comparison with other buble sorts}
a difference between the two lagnauges is going to be the way which they treat
fucntions. in c the construction of function which will return nothing or something
is going to have the same prototype, ada will actually will clearly segement these
two classes of functions. functions which are going to return nothing are going
to be referred to as \emph{procedures}, and functions which are going to return
something are going to be refferred to as \emph{sub-routines}. \par

a difference between c and ada is going to be the choice of terminating characters
for commands. in c the end of a construct such as a "if and else" statement will
need to be terminated withe a semi-colon, and in c the end of a "if and else"
statement is going to be terminated with a right parenthesis.\par

a similarity between c and ada implementation is going to be found in the
positioning of the functions in the file. in c this phenomenon is going to be
refferred to as forward declartion whereby, the functions (procedures) have to
be declared before the main body of code otherwise both the ada and c compilier
will complain that it doesn't kow what the following funciton is going to be. this
idea will also extend to variable decelartoin positioning as well, in c-89
variables will have to be decaled first before any commands are written, and in
ada variables and types will have to be declared before the \emph{begin} key word.\par

another similarity point between the ada and c implementation is going to be the
use of pointer manipulation in order to swap any element which is going to be greater
than the element which is going to be infront of it as both algorithms will
derefrence that memory location to get what is going to be stored their.\par

c and ada are going to be both strongly type languages. meaning that that a variable
or a data structure must be declared with a type before they're going to be used.
this idea will extend to function imports as well, the type of the imports must be
declared before use

\textbf{come back to this, i don't feel like writing it more}

furthermore, ada and c will both require you to import packages in order to be able
to do simple commands such as print a satement to the screen, and to use data
strucutres such as arrays. \par

\subsection{ada: reflection}
\textbf{come back to this as well}


\section{yacc and lex}


\subsection{reflection}
i really enjoyed the yacc and lex practical, although it was very time consuming
and painful to do. overal, it was a good experience to see how the basis of
a programming langauge are going to be constructed. furthermore, doing this
practical gave on appreciation of some of the fundemental concepts taught in my
computer science degrees.\par

a thing point of frustration while writign the programme is that the \emph{yytext}
variable was going to be a type of yystype which is going to be yacc's own datatype
for defining a string. therefore, durind compilitation i didn't think much about
it but, as soon as i ran the programme through yacc it produced segmentation faults.
since, i thought the variable type yystype was just a warning, and it was
going to be harmless i over looked the warning and that led into many hours of
trying to discover on why my programme was producing a segementation fault. i
would have wished the yacc compilier would have treated the data type mistmatch
as an actually error as other languages such as java would have. therefore,
since the compilier will allow access of invalid memory addresses this going to
make the yacc language less \emph{reliable} to write in. \par

yacc and lex code are both going to be segemented into sections, whereby the
first section is going to be the imports into the programme, and then after that
it's going to be either the yacc or lex definitions. the next section in the lex
file is going to be the tokenising rules and the corresponding code it is going
to send to yacc, in yacc the next section is going to be the grammar of the
language, and finally the last section is going to be the c functions which are
going to be associated with each file. due to the structured nature of the
programming language it made it easier to find area of interest i.e. if you want
to change the manner which lex tokenize strings you can jump to the middle of the
file. therefore, for this reason yacc and lex are going to adhere to the
\emph{structured programming principle} and as a consquence making the lnanguage
readable.\par

a good thing with yacc is that it was really easy to pick up as it was heavily
coupled to the c language. therefore, they was not that much to learn in the
yacc language except on how it would process its language grammars. \par

\section{scripting languages}


\subsection{scripting languages: reflection}

\subsubsection{bash:} although, i have used bash since starting my computer science
degree, and i use it in my daily navigation of the terminal environment i have
always find it hard to write. the reason why i think that bash scripting is hard
to write is becuase it doesn't follow the common conventions which i am accostumed
to with c, java, and python.for example, if you want to access a variable it's not
just enought to use its name like how you would in java, c and python, you will have
to have the variable name with a \$ then you can use the desired variable. furthermore,
if you want to declare a variable as somethign you will have to be cautious of your
spacing in bash you can't have spaces before or after the assignment of your variable,
in languages such as java, c, and python they don't really care on how much spaces you
will have. these just some of bash conventions which are different from the
languages which i am accoustmed too hence, while scripting in bash i will have
would have to be thinking of a lot more things than i would be typically doing
while programming hence fort his reason bash has low \emph{writability}. \par

in conjuction to the point discussed above, bash will have different methods to
be able to access specific commands. you can access a command by just typing
the name of the command, and with some of commands you will have to access them
using the back-tick (`) which adds another layer of thought  while scripting.
the question then becomes "am i accessing this command the right way", which is
typically not a question i would ask myself while writing other languages.
therefore, in this regards bash will lack \emph{regulairty}, and will add another
layer of diffuculty for \emph{writability}.


\subsubsection{ruby} out of the three scripting languages done, ruby was the
easiest one to write in because it's a close represention to the python scripting
language, and java script to some extent. therefore, given that i have had a
vast exprienc; in python in relation to industry projects, teaching, and
univeristy assignment the \emph{wrtiability} of ruby was far better than bash,
and perl.\par

furthermore, due to the close representation of ruby to other popular scripting
languages this makes ruby very easy to learn, and to understand what is being
conveyed code. additionally, the syntax of ruby is very simplisitic, and the
strucutre of the language closly represents what the programme's aim  is  going
to be. therefore, in this manner ruby also adheres to the \emph{simplicity}
programming principle.

\subsubsection{perl} in relation to the three scripting langauges i would place
perl as a middle child in relation to \emph{writiability}. perl borrows syntatic
constructs, and conventions from both the bash scripting language, and modern
popular scripting languages such as python and java script. for example perl
relays heavily on annoymous functions and calling functions within a function
as demonstrated in figure \ref{perl:demo} which is a design
patern which is heavily used in javascript, and a little bit in java. additionally,
perl only had fewer deviations from the conventional strucutre of modern day programming
for example with perl, it doesn't really matter on how many blanks you have after
an assignment which follows normal programming conventions. although, like bash
if you want to access a variable, the variable has to be preceeded by a dollar-sign
("\$"). therefore, perl borrows ideas from modern languages which i am accoustmed too
and borrow ideas from unix based scripting languages therefore, placing perl as
a middle child in relation to \emph{writability}.

\begin{figure}[!htp]
    \begin{problem}
        \begin{verbatim}
        #files wanted is a reference to the address of a function
        find(\&filesWanted, $searchPath)
        sub filesWanted{
                #code for your function
            }
        \end{verbatim}
    \end{problem}
    \caption{Demonstration of Perl's design strucutre}
    \label{Perl:demo}
\end{figure}

\section{Small-talk}

\subsection{Small-talk: Discussion}

\subsection{Small-talk: Reflection}

Constructing the conditional statements was the hardest part of the small talk
practical this is because small talk doesn't natively have if-else-if statements,
and only has if-else statemnets. To simulate the if-else-if statement nature of the
programme, Small-talk will require you to nest an if-else staement inside the else
clause of the parent if-else statement as demonsstraed in figure \ref{ST:IF_ELSE}.For,
this reason it made writing the Small-talk programme more diffucult as I had to
keep a consicious note on the location and the number of terminating square brackers
("]") hence, in this regards Small-talk will have low \emph{writability}. Additionally,
for this reason, it's diffucult to see the purpose of the if-else structure from firs
glance as compared to the native if-else-if statements found in C, Java and Python. To
be able to understand the structure it would required the programme to actually
carefully read the programme and in this regards Small-talk is going to have
low \emph{readability}.

\begin{figure}[!htp]
    \begin{problem}
        \begin{verbatim}
        <True condition> ifTrue: [ <statment ] ifFalse:
        [ <child if-else-statement> ]
        \end{verbatim}
    \end{problem}
    \caption{Small talk if-else-if statments}
    \label{ST:IF_ELSE}
\end{figure}

\section{C++}

\subsection{C++:Discussion}

\subsection{C++: Reflection}
Out of the covered languages in the Programmning languages assignment, this was
probably the most intutive, easiest, and most well rounded programming languages as
compared to the ones coverred in the unit. This is due to that C++ is very similar
to programming languges which I have experience namely Java and C. Therefore,
writing the C++ programme was a lot easier as it follows most of the programing
principles which I am accoustmed too. C++ is \emph{reliable} as the complier is
very helpful in it's error messages and wil tell you exactly what is wrong in
your code. Additionally, the C++ is almost the same as Java and C syntax hence,
it was easier for me read the C++ programmes and to know the funcion of a
porgramme therefore, in this regards C++ is \emph{syntacically} consistent with
other languages; C++ is \emph{regular} as all groups of conditions and constructs
are going to be the same throughout the language; and C++ has the same structure
as Java, and C therefore it's \emph{structually} consistent with those languages
and also the strucutre of the language represents it's function and purpose therefor
adhering to the \emph{structured programming} principle.\par

C++ is a language which encourages the use of streams, and most of its constructs
is written to be used in conjuction with streams. However, C++ doesn't enforce
the use of streams as it iwll allow you to do operations in the manner which you're
accoustemed too hence, making the barrier in learning C++ a lot smaller, as other
languages will force you in doing thing their way. Therefore, for that reason
C++ is a lot easier to learn and to pick up if you have prior exprience to Java,
and C.

\section{Prolog}

\subsection{Prolog: Discussion}
Throughout my experience of programming I have only dealt with imperative languages
hence, languages which you tell what is should do at each every single step.
Prolog is a logical language whereby it's paradigm is that the programmer is
going to specify the world's rules and the world's facts, and the language is
going to make conclusions based on those rules and facts. Prolog is going to be
based on the idea of a decision tree where it will use forward chaining and
backwards chaining to form conclusions. For this reason writing the Prologa


\subsection{Prolog: Reflection}

\section{Scheme}

\subsection{Scheme: Discussion}

\subsection{Scheme: Reflection}





\end{document}
